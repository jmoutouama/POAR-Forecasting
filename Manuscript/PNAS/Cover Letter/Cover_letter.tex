%%%%%%%%%%%%%%%%%%%%%%%%%%%%%%%%%%%%%%%%%
% Long letter BUAA Version
% Version 1.0 (2024-05-18)

% This template was revised by Wang Jianghai(wang_jianghai@buaa.edu.cn) based on Fanchao Chen (chenfc@fudan.edu.cn).
% This template originates from:
% https://www.LaTeXTemplates.com

%%%%%%%%%%%%%%%%%%%%%%%%%%%%%%%%%%%%%%%%%
%----------------------------------------------------------------------------------------
%	PACKAGES AND OTHER DOCUMENT CONFIGURATIONS
%----------------------------------------------------------------------------------------

\documentclass{article}

\usepackage{charter} % Use the Charter font

\usepackage[
	a4paper, % Paper size
	top=1in, % Top margin
	bottom=1in, % Bottom margin
	left=1in, % Left margin
	right=1in, % Right margin
	%showframe % Uncomment to show frames around the margins for debugging purposes
]{geometry}

\setlength{\parindent}{0pt} % Paragraph indentation
\setlength{\parskip}{1em} % Vertical space between paragraphs

\usepackage{graphicx} % Required for including images

\usepackage{fancyhdr} % Required for customizing headers and footers

\fancypagestyle{firstpage}{%
	\fancyhf{} % Clear default headers/footers
	\renewcommand{\headrulewidth}{0pt} % No header rule
	\renewcommand{\footrulewidth}{1pt} % Footer rule thickness
}

\fancypagestyle{subsequentpages}{%
	\fancyhf{} % Clear default headers/footers
	\renewcommand{\headrulewidth}{1pt} % Header rule thickness
	\renewcommand{\footrulewidth}{1pt} % Footer rule thickness
}

\AtBeginDocument{\thispagestyle{firstpage}} % Use the first page headers/footers style on the first page
\pagestyle{subsequentpages} % Use the subsequent pages headers/footers style on subsequent pages

%----------------------------------------------------------------------------------------

\begin{document}

%----------------------------------------------------------------------------------------
%	FIRST PAGE HEADER
%----------------------------------------------------------------------------------------

\includegraphics[width=0.4\textwidth]{Rice_Logo_280_Blue.pdf} % Logo

\vspace{-1em} % Pull the rule closer to the logo

\rule{\linewidth}{1pt} % Horizontal rule

\bigskip\bigskip % Vertical whitespace

%----------------------------------------------------------------------------------------
%	YOUR NAME AND CONTACT INFORMATION
%----------------------------------------------------------------------------------------

%\hfill
%\begin{tabular}{l @{}}
%\hfill \today \bigskip\\ % Date
%\hfill Name \\
%\hfill Email: xxx@buaa.edu.cn \\
%\hfill 37 Xueyuan Road, Haidian District,\\
%\hfill Beijing, P.R.China, 100191 \\ % Address
%\end{tabular}

\bigskip % Vertical whitespace

%----------------------------------------------------------------------------------------
%	ADDRESSEE AND GREETING
%----------------------------------------------------------------------------------------

\begin{tabular}{@{} l}
	Dr May R. Berenbaum \\
	Editor-in-chief \\
	\textit{PNAS}
\end{tabular}

\bigskip % Vertical whitespace

Dear Dr May R. Berenbaum,

\bigskip % Vertical whitespace

%----------------------------------------------------------------------------------------
%	LETTER CONTENT
%----------------------------------------------------------------------------------------

We would like to submit our manuscript titled “Forecasting Range Shifts of a Dioecious Plant Species Under Climate Change” for publication in  \textit{
Proceedings of the National Academy of Sciences}.

Recent studies on dioecious plants suggest that future climate change may favor male-biased sex ratios. However, the vast majority of models used to forecast population viability and range shifts in response to climate change do not account for sex ratio bias, and thus do not consider the potential for females and males to differ in their sensitivity to climate drivers.

In this study, we  used a unique dataset on a dioecious plant (Texas bluegrass) collected from a common garden experiment distributed over a dramatic environmental gradient in the southern Great Plains, USA combined with Bayesian statistics and mathematical models to investigate whether accounting for the complexity of sex structure affects predictions of dioecious species’ responses to climate change. Our findings demonstrate that considering only one sex can lead to an underestimation of the impact of climate change on dioecious species, particularly in regions of their range that are biased toward one sex.

Our results are original and lie at the interface of population biology and conservation biology, making them well suited for \textit{PNAS}.

We have opted out of double-blind review because this paper builds upon our previous work (one component of our modeling framework is published elsewhere), so our identities could be easily deduced. Our identities are also evident from our GitHub repository, where we direct reviewers to find all data and code necessary to reproduce our analyses.

\bigskip % Vertical whitespace

Sincerely yours,

\vspace{20pt} % Vertical whitespace


Jacob Moutouama \\
Aldo Compagnoni \\
Tom Miller


\end{document}