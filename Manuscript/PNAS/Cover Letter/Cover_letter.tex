%%%%%%%%%%%%%%%%%%%%%%%%%%%%%%%%%%%%%%%%%
% Long letter BUAA Version
% Version 1.0 (2024-05-18)

% This template was revised by Wang Jianghai(wang_jianghai@buaa.edu.cn) based on Fanchao Chen (chenfc@fudan.edu.cn).
% This template originates from:
% https://www.LaTeXTemplates.com

%%%%%%%%%%%%%%%%%%%%%%%%%%%%%%%%%%%%%%%%%
%----------------------------------------------------------------------------------------
%	PACKAGES AND OTHER DOCUMENT CONFIGURATIONS
%----------------------------------------------------------------------------------------

\documentclass{article}

\usepackage{charter} % Use the Charter font

\usepackage[
	a4paper, % Paper size
	top=1in, % Top margin
	bottom=1in, % Bottom margin
	left=1in, % Left margin
	right=1in, % Right margin
	%showframe % Uncomment to show frames around the margins for debugging purposes
]{geometry}

\setlength{\parindent}{0pt} % Paragraph indentation
\setlength{\parskip}{1em} % Vertical space between paragraphs

\usepackage{graphicx} % Required for including images

\usepackage{fancyhdr} % Required for customizing headers and footers

\fancypagestyle{firstpage}{%
	\fancyhf{} % Clear default headers/footers
	\renewcommand{\headrulewidth}{0pt} % No header rule
	\renewcommand{\footrulewidth}{1pt} % Footer rule thickness
}

\fancypagestyle{subsequentpages}{%
	\fancyhf{} % Clear default headers/footers
	\renewcommand{\headrulewidth}{1pt} % Header rule thickness
	\renewcommand{\footrulewidth}{1pt} % Footer rule thickness
}

\AtBeginDocument{\thispagestyle{firstpage}} % Use the first page headers/footers style on the first page
\pagestyle{subsequentpages} % Use the subsequent pages headers/footers style on subsequent pages

%----------------------------------------------------------------------------------------

\begin{document}

%----------------------------------------------------------------------------------------
%	FIRST PAGE HEADER
%----------------------------------------------------------------------------------------

\includegraphics[width=0.4\textwidth]{Rice_Logo_280_Blue.pdf} % Logo

\vspace{-1em} % Pull the rule closer to the logo

\rule{\linewidth}{1pt} % Horizontal rule

\bigskip\bigskip % Vertical whitespace

%----------------------------------------------------------------------------------------
%	YOUR NAME AND CONTACT INFORMATION
%----------------------------------------------------------------------------------------

%\hfill
%\begin{tabular}{l @{}}
%\hfill \today \bigskip\\ % Date
%\hfill Name \\
%\hfill Email: xxx@buaa.edu.cn \\
%\hfill 37 Xueyuan Road, Haidian District,\\
%\hfill Beijing, P.R.China, 100191 \\ % Address
%\end{tabular}

\bigskip % Vertical whitespace

%----------------------------------------------------------------------------------------
%	ADDRESSEE AND GREETING
%----------------------------------------------------------------------------------------

\begin{tabular}{@{} l}
	Dr May R. Berenbaum \\
	Editor-in-chief \\
	\textit{PNAS}
\end{tabular}

\bigskip % Vertical whitespace

Dear Dr May R. Berenbaum,

\bigskip % Vertical whitespace

%----------------------------------------------------------------------------------------
%	LETTER CONTENT
%----------------------------------------------------------------------------------------

We are pleased to submit our manuscript titled “Forecasting range shifts of dioecious plants under climate change” for consideration of publication in  \textit{
Proceedings of the National Academy of Sciences}.

Forecasting range shifts of plants and animals under climate change is among the most pressing challenges at the interface of science and society. 
For species with separate sexes (most animals and many plants), accumulating evidence suggests that females and males may differ in their sensitivity to climate change. 
Yet, the vast majority of models used to forecast population viability and range shifts do not account for sex structure, and thus do not consider the potential for females and males to differ in their sensitivity to climate drivers.

We used models and experiments to show, for the first time, how sex structure and sexual niche differentiation influence forecasts for range shifts under climate change. 
Using a combination of geographically distributed common garden experiments with dioecious plants, hierarchical Bayesian parameter estimation, and two-sex demographic modeling, our findings demonstrate that considering only one sex can lead to an underestimation of the impact of climate change on dioecious species.

This work demonstrates how ecological theory and experiments can be leveraged to build quantitative forecasts about biogeographic patterns while accounting for important sources of uncertainty. 
Given the current concern about how species will respond to rapid climate change, our results are timely and have implications for species of conservation concern.
We think our work is well-suited to the broad audience of \textit{PNAS}.

We have opted out of double-blind review because this paper builds upon our previous work, so our identities could be easily deduced. Our identities are also evident from our GitHub repository, where we direct reviewers to find all data and code necessary to reproduce our analyses.

\bigskip % Vertical whitespace

Sincerely yours,

\vspace{20pt} % Vertical whitespace


Jacob Moutouama \\
Aldo Compagnoni \\
Tom Miller


\end{document}